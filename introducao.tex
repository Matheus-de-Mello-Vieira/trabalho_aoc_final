ARM é a arquitetura mais comum em dispositivos embutidos, surgiu no mesmo ano que MIPS (1985). Sua versão original usa palavras de 32 bits, a mais recente usa de 64 bits. Quando foi inventada, sua sigla significava "Acorn RISC Machines" e era concebida para computadores pessoais, porém, atualmente significa "Advanced RISC Machines" e é largamente utilizado em sistemas embarcados e dispositivos móveis. Ela foi projetada para permitir implementação muito pequenas e com alta performasse. Isso se dá pela sua simplicidade e tem como consequência o baixo custo energético de seus processadores. Diferentes versões de ARM são licenciadas e produzidas por várias companhias, utilizada em servidores, computadores pessoais, dispositivos móveis e etc.

O ARM é um "Reduced Instruction Set Computer (RISC)", e como tal, possui os seguintes recursos:
\begin{itemize}
	\item um longo arquivo uniforme de registro
	\item uma arquitetura de load/store, onde o processamento de dados ocorrem apenas nos conteúdos dos registradores e não na memória diretamente
	\item modo de endereçamento simples
	\item campos de instruções uniformes e com tamanho fixo
\end{itemize}

Além disso, ARM também oferece:

\begin{itemize}
	\item controle sobre a ULA e shifter
	\item modo de endereçamento com auto-incremento e auto-decremento
	\item carregamento/armazenamento de múltiplas instruções
	\item execução condicional
\end{itemize}