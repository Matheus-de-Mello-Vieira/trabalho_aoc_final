\section{Modos de operações}

A arquitetura ARM suporta sete modos de processamentos, mostrados na seguinte tabela:

<tabela>

Os mudança de modo pode ser feito por meio de controle de software, ou pode ser causado por interrupções externas ou processamento de exceção. A maioria das aplicações são executadas no User mode, que é desprivilegiado, que não pode mudar para outro modo de operação sem gerar uma exceção ou acessar recursos do sistema protegidos. Os demais modos são privilegiados, que, exceto o System mode, são acessados quando uma excessão especifica acontece, possuindo registradores adicionais. O System mode não é acessado por meio de qualquer exceção e possuem exatamente os mesmos registradores que o User mode. Porem, é um modo privilegiado e não possui as restrições do User mode. Ele serve para operar tarefas de sistema que precisa de acesso ao recursos de sistema.

% Table generated by Excel2LaTeX from sheet 'Planilha1'
\begin{table}[h]
	\centering
	\begin{tabular}{l|l}
		Nome  & Descrição \\ \hline
		User  & modo normal de execução de programa \\
		FIQ   & suporta transferência de dados em alta velocidade ou processo de canal \\
		Interrupt & Usado para lidar com interrupções \\
		Supervisor & Um modo protegido para operar o sistema \\
		Abort & Implementa uma memória virtual e/ou proteção de memória \\
		Undefined & suporta emulação de co-processadores via software \\
		System & modo User com privilégios \\
	\end{tabular}%
	\label{tab:addlabel}%
\end{table}%
