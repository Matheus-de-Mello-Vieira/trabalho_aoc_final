\section{Instruções ARM}

as instruções ARM podem ser divididas em 6 classes:
\begin{enumerate}
	\item Instruções de ramificação
	\item Instruções de processamento de dados (A1-6)
	\item Instruções de transferência de registrador de status
	\item Instruções de load/store
	\item Instruções de co-processamento
	\item Instruções de gerar excessões
\end{enumerate}

Neste trabalho serão abordado apenas os quatro primeiros, já que o quinto e o sexto fogem do escopo deste trabalho.

A maioria das instruções de processamento de dados e de co-processamento podem ser atualizados para mudar o seu code flag (Negativo, zero, Carry, Verflow)

Quase todas as instruções contem um campo de 4 bit de condição. Um valor desse campo especifica que a instrução é executada incondicionalmente.

Os próximos 14 valores especificam uma execução condicional, que quando verdadeira, o comando é executada, são eles:
1) teste para igualdade e desigualdade
2) teste para $<$, $<=$, $>$ e $=>$ (para aritmética com sinal e sem sinal)
3) cada code flag

O decimo sexto valor e´usada em poucas instruções e não permite uma execução condicional.

\subsection{Instruções de ramificação}

\label{instrucao_ramificacao}

uma instrução de ramificação possui um offset sem sinal de 24 bits, permitindo avançar ou recuar em 32MB.

Existe a opção BL (branch and link) que preserva o endereço da instrção depois da ramificação no registrador 14, isso permite chamada de sub-rotinas.

\subsection{Instruções de processamento de dados}

Instruções de processamento estão divididas em 4 categorias:

\begin{itemize}
	\item instruções logica/aritméticas
	\item instruções de comparações
	\item instruções de multiplicações
	\item contar 0s à esquerda
\end{itemize}

Instruções de carga/armazenamento

Existem 3 tipos de instruções desse tipo:

\begin{itemize}
	\item carregar e armazenar um registrador
	\item carregar e armazenar múltiplos registradores
	\item trocar conteúdos do registrador e da memória
\end{itemize}

% Table generated by Excel2LaTeX from sheet 'Planilha1'
\begin{table}[h]
	\begin{tabular}{c|c|c|c}
		& \textbf{Nome da instrução} & \textbf{ARM}& \textbf{MIPS} \\\hline
		\multirow{14}[0]{*}{Registrador-registrador} & Add   & add   & addu, addiu \\
		& Add(trap if overflow) & adds; swivs & add \\
		& Subtract & sub   & subu \\
		& Subtract(trap if overflow) & sub; swivs & sub \\
		& Multiply & mul   & mult, multu \\
		& Divide & -     & div, divu \\
		& And   & and   & and \\
		&  Or   & orr   & or \\
		& Xor   & eor   & xor \\
		& Load high part register & -     & lui \\
		& Shift left logical & lsl   & sllv, sll \\
		& Shift right logical & lsr   & srlv, srl \\
		& Shift right arithmetic & asr   & srav, sra \\
		& Compare & cmp, cmn, tst, teq & slt/i, slt/iu \\\hline
		\multirow{10}[0]{*}{Transferência de dados} & Load byte signed & ldrsb & lb \\
		& Load byte unsigned & ldrb  & lbu \\
		& Load halfword signed & ldrsh & lh \\
		& Load halfword unsigned & ldrh  & lhu \\
		& Load word & ldr   & lw \\
		& Store byte & strb  & sb \\
		& Store halfword & strh  & sh \\
		& Store word & str   & sw \\
		& Read, write special registers & mrs, msr & move \\
		& Atomic Exchange & swp, swpb & ll;sc \\
	\end{tabular}%
\end{table}%

